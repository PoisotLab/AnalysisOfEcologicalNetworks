\chapter{Introduction\label{chap:introduction}}

When we first attempted to synthetize the literature on \emph{how to best
analyse ecological networks?} \parencite{DelBesBri18}, two things became clear.
First, an increasing number of fields in ecology were using networks as a
formalism to represent and analyse their data. Second, methods were multiplying
at an alarming rate, and there was a great degree of methodological confusion
about what to use, how to interpret it, and when not to use it. In short, the
field of ecological network analysis was long due for a textbook.

\section{Why networks?}

\section{What this book is about}

\subsection{The structure of ecological networks}

\subsection{Best practices in ecological networks research}

\subsection{Programming}

\section{What this book is not about}

\subsection{The ecology of ecological networks}

\subsection{The foundation of network theory}

There are a number of fantastic textbooks on network science, where the
mathematical foundations of this field are laid out in a clear and comprehensive
way. One of the best ways to share the excitement about why the mathematics of
networks theory are worth looking into is to read a short, and very accessible,
perspective by \textcite{Str01}. Should this whet your appetite, there are
multiple resources that are suitable to get a deeper education in graph theory.
Two outstanding references are the now classic \textcite{New10}, and the more
recent \textcite{Bar16}. For readers seeking more introductory material, we
recommed the short and very accessible volume by \textcite{Cha85}, as well as
\textcite{Wes01}.

But fear not -- this book is not about mathematics. The measures of ecological
network structure we will discuss have are grounded in mathematics, but the
point is \emph{not} to go into much details into it. There will be a \emph{fair}
amount of mathematics, probably as much as during an advanced class in
statistics (which is to say just enough to fill the average ecologist with
apprehension, but not quite with dread), but it will not be the point of the
material. This is merely something that comes with the analysis of ecological
networks. You will realize through the different chapters that there is only a
very small quantity of graph theory we need to know to do good ecology; this
will have the highly desirable side effect to keep any mathematical posturing to
a minimum.

\subsection{Programming}

\section{How to read this book}

\subsection{The importance of primary literature}

\subsection{The importance of experimenting with code}

This books involves \emph{a lot} of computer code; in fact, it is mostly about
writing computer code to analyse ecological networks. Although we rely
extensively on a software package \parencite{PoiBelHoe19} to do this, there will
be many applications where writing our own functions, or writing our own
scripts, will be necessary. Copying and pasting this code, and running it
exactly as presented here, is a terrible idea. Instead, we encourage readers to
\emph{adapt} the code to their own interests, uses, and questions.

How does one adapts code? \textbf{tk}

\subsection{The importance of consistent notation}

Throughout this book, we strive to use notation that remains as constant as
possible. Networks are represented by capital letters, with the exact letter
used representing some additional information about the type of network.
Unipartite networks are $U$, bipartite networks are $B$, probabilistic networks
are $P$, random networks are $R$, quantitative networks are $G$ (only because
$Q$ is used for modularity already), and a network of an unspecified type is
$N$. When there is a collection of multiple networks, we use the same letter in
bold face -- for example, we can note that a function generates random networks
from a bipartite network using $f(B) = \mathbf{R}$.

To refer to a specific interaction, we will use the notation $N_{ij}$, which
represents an interaction \emph{from} species $i$ \emph{to} species $j$. In some
situations, we will also use $N(i,j)$ to represent the same information. The
notation $N_{i\cdot}$ and $N_{\cdot j}$ indicate, respectively, the set of
species with which $i$ interacts, and the set of species that interact with $j$.
For example, the number of predators of species 4 in a unipartite food web is
$\|U_{\cdot 4}\|$.
