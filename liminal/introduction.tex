\chapter{Introduction\label{chap:introduction}}

\parencite{DelmBess18}

\section{Why networks?}

\section{What this book is about}

\subsection{The structure of ecological networks}

\subsection{Best practices in ecological networks research}

\subsection{Programming}

\section{What this book is not about}

\subsection{The ecology of ecological networks}

\subsection{The foundation of network theory}

There are a number of fantastic textbooks on network science, where the
mathematics foundations of this field are laid out in a clear and comprehensive
way.

\subsection{Programming}

\section{How to read this book}

\subsection{The importance of primary literature}

\subsection{The importance of experimenting with code}

This books involves \emph{a lot} of computer code; in fact, it is mostly about
writing computer code to analyse ecological networks. Although we rely
extensively on a software package \parencite{poisot_ecologicalnetworks.jl_2019}
to do this, there will be many applications where writing our own functions, or
writing our own scripts, will be necessary. Copying and pasting this code, and
running it exactly as presented here, is a terrible idea. Instead, we encourage
readers to \emph{adapt} the code to their own interests, uses, and questions.

How does one adapts code? \textbf{tk}
